\documentclass[a4paper,10pt]{article}
\usepackage[utf8]{inputenc}

\usepackage{amsmath}
\usepackage{amsfonts}
\usepackage{amsthm}
\usepackage{amssymb}
\usepackage{color}
\usepackage{tikz}
\usepackage{manfnt}
\usepackage{todonotes}
\usepackage{textcomp}
\usetikzlibrary{decorations.markings}
\usepackage[all]{xy}
%\usepackage[nottoc,notlot,notlof]{tocbibind}
\tikzset{->-/.style={decoration={
  markings,
  mark=at position .5 with {\arrow{>}}},postaction={decorate}}}
%opening



% or if manfnt is unavailable, uncomment the next two lines
%\font\manual=manfnt
%\def\dbend{{\manual\char127}} % dangerous bend sign

%%
% This macro header is what controls the ``dangerous bend''
% paragraph
%%

% Danger, Will Robinson!
\newenvironment{danger}{\medbreak\noindent\hangindent=2pc\hangafter=-2%
  \clubpenalty=10000%
  \hbox to0pt{\hskip-\hangindent\dbend\hfill}\small\ignorespaces}%
  {\medbreak\par}

% Danger! Danger!
\newenvironment{ddanger}{\medbreak\noindent\hangindent=3pc\hangafter=-2%
  \clubpenalty=10000%
  \hbox to0pt{\hskip-\hangindent\dbend\kern2pt\dbend\hfill}\small\ignorespaces}%
  {\medbreak\par}

%\mathrel{\mathop:}=

\newcommand{\set}[1]{\left\{ #1 \right\}}
\newcommand{\norm}[1] {\| #1 \|}
\newcommand{\Hom}[3]{\mbox{Hom}_{#1} \left( #2, #3 \right)}
\newcommand{\inner}[2]{\langle #1, #2 \rangle}
\newcommand{\ceil}[1]{\lceil #1 \rceil}
\newcommand{\floor}[1]{\lfloor #1 \rfloor}
\newcommand{\mbf}[1]{\mathbf{#1}}
\newcommand{\mbb}[1]{\mathbb{#1}}
\newcommand{\R}{\mathbb{R}}
\newcommand{\C}{\mathbb{C}}
\newcommand{\Z}{\mathbb{Z}}
\newcommand{\Q}{\mathbb{Q}}
\newcommand{\N}{\mathbb{N}}
\newcommand{\GL}{\mathrm{GL}}
\newcommand{\mcal}[1]{\mathcal{#1}}
\newcommand{\ang}[1]{\langle #1 \rangle}
\newcommand{\dfn}{\mathrel{\mathop:}=}
\newcommand{\supp}[1]{\mathrm{Supp}(#1)}
\newcommand{\path}[3]{\mbox{path}_{#1}{(#2,#3)}}
\newcommand{\wmu}{\widetilde{\mu}}
\newcommand{\ex}[1]{\exp \left( #1 \right)}
\newcommand{\LscLz}[2]{\mathrm{Lip}^{\mathrm{ls}}(#1,#2)}
\newcommand{\mfrak}[1]{\mathfrak{#1}}
\newcommand{\BredR}[2]{\mathrm{Mod}_{\mfrak{#1}}$-$#2}
\newcommand{\LBred}[2]{#1$-$\mathrm{Mod}_{\mfrak{#2}}}
\newcommand{\mor}[2]{\mathrm{mor}(#1,#2)}


\theoremstyle{plain}% default
\newtheorem{thm}{Theorem}[section]
\newtheorem{lem}[thm]{Lemma}
\newtheorem{prop}[thm]{Proposition}
\newtheorem{cor}[thm]{Corollary}
\newtheorem{dt}[thm]{Definition and Theorem}

%No numbering
\newtheorem*{thmR}{Theorem}
\newtheorem*{lemR}{Lemma}
\newtheorem*{propR}{Proposition}
\newtheorem*{corR}{Corollary}
\newtheorem*{dtR}{Definition and Theorem}

\theoremstyle{definition}
\newtheorem{defn}[thm]{Definition}
\newtheorem{exmp}[thm]{Example}
\newtheorem{exmps}[thm]{Examples}
\newtheorem{qu}[thm]{Question}
\newtheorem{dprop}[thm]{Definition and Proposition}

\newtheorem*{defnR}{Definition}
\newtheorem*{exmpR}{Example}
\newtheorem*{exmpsR}{Examples}
\newtheorem*{quR}{Question}
\newtheorem*{dpropR}{Definition and Proposition}

\theoremstyle{remark}
\newtheorem{rem}[thm]{Remark}

%\setlength{\parindent}{0in}

\begin{document}
\section{Automorphisms of Houghton's Group}
We aim to find the automorphisms of Houghton's group $H_n$ which acts on the set $X_n$ ($n$ branches of $\N$). Throughout we shall consider $n \ge 2$ as $H_1 \cong FSym$ (the group of all permutations with finite support)
and $Aut(FSym) \cong Sym$ (the group of all possible permutations). We will also omit $X_n$ when it is clear from the context that it is the set being acted on by $FSym$  or
$Sym$. Note $FSym(X_n) \subseteq H_n \subseteq Sym(X_n)$.
\begin{thm} $Aut(H_n)$ has solvable conjugacy problem.
\end{thm}
This has not yet been achieved. First we will calculate the structure of $Aut(H_n)$. We will prove the following.
\begin{lem}\label{1}
$Aut(H_n) \cong N_{Sym}H_n.$
\end{lem}
\begin{lem}\label{2}
$N_{Sym}H_n = S_n \ltimes H_n$.
\end{lem}
An important point to note is that finite extensions do not pass conjugacy solvability across \cite{Gorjaga1975}. This means that despite the fact that $H_n$
has solvable conjugacy problem \cite{ConjHou} and that $S_n \ltimes H_n$ is a finite extension of $H_n$ we do not automatically
have that $Aut(H_n)$ has solvable conjugacy problem.

Our first steps are towards Lemma \ref{1}. Using that every automorphism  of FSym is conjugation by $\rho \in Sym$ and Lemma \ref{3} below,
\bigskip
\bigskip
\bigskip
\bigskip
\bigskip
\begin{align*}   \textrm{Meaning} \;\;  \rho \sigma |_{FSym} \rho^{-1} &=id_{FSym}\;\; \textrm{and so}\;\; \rho \sigma \rho^{-1} = id_{H_n}
     \end{align*}
We will then have that every $\sigma \in Aut(H_n)$ can be achieved through conjugation by $\rho \in Sym$ so that there is an epimorphism from $N_{Sym}H_n$ to $Aut(H_n)$.
\begin{lem}\label{3} Let $\sigma \in Aut(H_n)$. If $\sigma |_{FSym}=id_{FSym}$ then $\sigma=id_{H_n}$
\end{lem}
We consider the action of the standard generators $g_i$ after such an automorphism $\sigma$. To simplify the action of $g_i$ we identify it's support (2 branches of $X_n$) with $\Z$.
\begin{proof} Let $g$ be a standard generator and $\sigma \in Aut(H_n)$ with $\sigma |_{FSym}=id_{FSym}$.
\begin{align*}(\sigma g)^{-1} (i\; i+1)(\sigma g) &= \sigma(g(i\; i+1))=\sigma(i-1\; i)=(i-1\; i)\\
(\sigma g)^{-1} (i\; i+1)(\sigma g) &=(\sigma g(i) \; \sigma g(i+1))
\end{align*}
Therefore $\sigma g(i) \in \{ i-1,\; i\}$. Running the argument for the transposition $(i-1 \; i)$ shows $\sigma g(i) \in \{i-2, i-1\}$ and so $\sigma g(i) = i-1$.

As this argument holds for all $i \in \Z$ (the two branches our given generator acts on) we have that on the branches $g$ acts on, $\sigma g(i)$ acts in the same way.

By checking that $supp(\sigma (g))=supp(g)$ we will have proved our claim for the generators of $H_n$ and so the claim will extend to any element of $H_n$.
This can be done by noticing that for the argument above if we consider those points fixed by $g$ (i.e. not on the copy of $\Z$ that $g$ translates) then $\sigma g$ fixes all
such finite permutations. To show that $\sigma g$ does not translate any other branch we note that for all $h \in H_n$ such that $|h|<\infty$ and $supp(h) \cap supp(g) = \emptyset$ we have that $h^g = h$ and the following holds.
\begin{align*}
&h^g = h \Rightarrow \sigma(h^g)=\sigma(h) \Rightarrow \sigma(g^{-1})\sigma(h)\sigma(g)=\sigma(h) \Rightarrow \sigma g \in C_{H_n} (\sigma h)
\end{align*}
As $\sigma$ is an automorphism $\sigma h$ cannot be of infinite order and so $\sigma g$ cannot act on any other branch.
\end{proof}
This means we have a epimorphism $m: N_{Sym}H_n \xrightarrow\!\!\!\!\to Aut(H_n)$. For the following argument we will need that $C_{Sym}(H_n)$ is trivial. It is enough to show that $C_{Sym}(FSym)$ is trivial.
Assume there is a $\rho \ne 1$ in Sym such that $\forall \; g \in FSym, \rho g=g\rho$. Then there is some $i \in supp(\rho)$. So pick $j \not\in \{\rho(i), i\}$. Now $(i, j)^{\rho}=(\rho(i), \rho(j))=(i, j)$ gives us our contradiction
as $\rho(i)\ne i$ or $j$.
\begin{lem} m is injective.
\end{lem}
\begin{proof}
Consider $\rho_1, \rho_2 \in Sym$ such that $\forall g \in H_n\; \rho_1g\rho_1^{-1}=\rho_2^{-1}g\rho_2$. Clearly $\rho_2\rho_1g =g\rho_1\rho_2$. Therefore $\rho_1=\rho_2^{-1}$ as $C_{Sym}(H_n)$ is trivial.
This means given 2 conjugates in Sym which define the same automorphism of $H_n$ the conjugates must be equal.
\end{proof}
Now we move onto Lemma \ref{2}. We note that $S_n$ acts on $X_n$ by permuting its branches. From this $H_n \unlhd S_nH_n$ and clearly such an action has trivial intersection with $H_n$ meaning $N_{Sym}H_n \supseteq
S_n \ltimes H_n$.

\bibliographystyle{amsalpha}
\bibliography{bibliography}
\end{document}